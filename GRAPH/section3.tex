\section{Connectivity}

\subsection{Connectivity}



A vertex cut of G is a subset V' of V such that G - V' is disconnected. A
k-vertex cut is a vertex cut of k elements. A complete graph has no vertex
cut; in fact, the only graphs which do not have vertex· cuts are those that
contain complete graphs as spanning subgraphs. If G has at Ie.ast one pair of
distinct nonadjacent vertices, the connectivity K(G) of G is the minimum k
for which G has a k-vertex cut; otherwise, we define K(G) to be v-·I. Thus
K (G) = 0 if' G is either trivial or disconnected. G is said to be k -connected if
K(G) >- k. All nontrivial connected graphs are I-connected.
Recall that an edge cut of G is a subset of E of the form [5, S], where 5 is
a nonempty proper subset of V. A k-edge cut is an edge cut of k elements.
If G is nontrivial and E' is an edge cut of G, then G - E' is disconnected; we
then define the edge connectivity K'(G)ofG to be the minimum k for which
G has a k-edge cut. If G is trivial, K'(G)is'defined to be zero. Thus K'(G) = 0
if G is either trivial or disconnected, and K '(G) = I ifG is a connected graph
with a cut edge. G is said to be k-edge-connected if K'(G) >- k. All nontrivial
connected graphs are l-edge-connected.