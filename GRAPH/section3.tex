\section{Connectivity}

\subsection{Connectivity}

\begin{dfn}[Connectivity] 
    Connectivity
    \begin{itemize}
        \item A vertex cut : $G-V'$가 연결되지않은 그래프인 $V$의 부분 집합 $V'$를 정점 절단(A vertex cut) 이라고한다.
        \item k-vertex cut : k개의 원소를 가진 정점 절단. 완전 그래프(complete graph)는 정점 절단을 가지지 않는다.
        \item 스패닝 서브 그래프로서 완전 그래프를 가지는 그래프는 정점 절단을 가지지 않는다.
        \item  connectivity $\kappa(G)$ : 그래프 $G$가 가지는 $k-vertex cut$의 최소값 $k$를 $\kappa(G)$라고 한다.그래프 $G$가 trivial이거나 연결되지않은 그래프일때 $\kappa(G) = 0$이다.
        \item $k-connected$ :  $\kappa(G) \ge k$일때 그래프 $G$는 $k-connected$이다. 
        \item 모든 nontrivial connected graph는 $1-connected$이다.
    \end{itemize}
\end{dfn}

\begin{dfn}[Edge connectivity]
    Edge connectivity 
    \begin{itemize}
        \item k-edge cut : k개의 원소를 가지는 간선 절단(edge cut).
        \item Edge connectivity $\kappa'(G)$ : nontrivial 그래프 $G$의 k-edge cut $E'$ $G-E'$는 연결되어있지않다. $k-edge cut$를 가지는 $G$의 최소한의 $k$를 $\kappa'(G)$라 표현한다. $G$가 trivial이거나 연결되지않은 그래프 일때, $\kappa'(G) = 0$이다.
        \item $\kappa(G) \ge k$일때 $G$는 k-edge-connected이다.
        \item 그래프 $G$가 연결된 그래프일때  $\kappa'(G)= 1$이고 모든 nontrivial connected 그래프는 1-edge-connected이다.
    \end{itemize}
\end{dfn}


\subsection{BLOCKS}

\begin{dfn}[Blocks]
    Block : 절단 정점들이 존재 하지않는 연결된 그래프를 block이라한다.
    적어도 세개 이상의 정점을 가진 모든 블록은 2-connected이다.
    그래프의 블록은 최대로 블록의 성질을 가질수있는 상태이다.(대충 블록이 또 블록으로 쪼개지는 경우는 생각 안한다는것.)
\end{dfn}

\begin{theorem}
    
\end{theorem}