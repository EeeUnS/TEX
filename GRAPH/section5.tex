\section{Matchings}

\subsection{Matchings}

\begin{dfn}[Matching]

    $E$의 부분 집합 $M$이 각 원소가 link이고 그래프 $G$에서 서로 인접하지않으면 G의 매칭(matching)이라고 한다.

    
    
    \begin{itemize}
        \item M의 간선의 양끝을 M에 일치되어있다(be matched under M)라고 한다.
        \item saturated : M의 간선이 정점 v에 인접할때 v를 포화되었다 (be Saturated)라 하고, 매칭 M이 정점 $v$를 포화시킬때, $v$를 M-saturated 되었다(be M-saturated) 라고한다. 반대는 M-unsaturated라 한다.
        \item 모든 정점이 M-saturated 되었을때, M을 완전(perfect)이라고 한다.
        \item M-alternating path : $E/M$과 M의 간선을 교대로 선택한 path
        \item M-augmenting path : 처음과 끝이 M-unsaturated인 M-alternating path
    \end{itemize}


\end{dfn}

%5.1
\begin{theorem} 매칭 M은 맥시멈 매칭이다와 G가 M-argumenting path를 가지지 않는다 는 필요충분 조건이다.
    
\end{theorem}

\begin{proof}
    
\end{proof}

\subsubsection{}
%5.1.1
(a) :
(b) : 
$K_2n$ : $\prod_{k=1}^{n} (2k-1)$

$K_{n,n} = n!$
\subsubsection{}
%5.1.2
case 1: $V$ 갯수가 홀수인경우 퍼펙트매칭을 가질 수 없음.
case 2: $V$ 갯수가 짝수인데 홀수개인 차수를 가지는 정점이 있는경우
\subsubsection{}
%5.1.3 삼각형
\subsubsection{}
%5.1.4

\subsubsection{}
%5.1.5