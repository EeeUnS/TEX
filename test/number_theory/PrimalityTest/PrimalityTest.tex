\section{소수 판별법}

대표적인 방법

 1부터 $\sqrt{n}$ 자연수까지 모두 나눠본후 나눠지는 수가없을시에 그수는 소수이다. 2일때는 소수라 하고 짝수로 나누는 경우는 없애도 상관없다.
$O(\sqrt{n})$

에라토스테네스의 체는 특정 $n$이 소수인지 판단하는것과는 무관하므로 제외한다..


이글의 연장선상인 얘기이다.

하나의 값 $n$을 놓고 이값이 소수인지 아닌지 보려면 현재 $O(\sqrt{n})$까지 만큼 비교해 보아야 확정적으로 알수있다.

엄청난 크기의 소수를 구하기 위해서 $O(\sqrt{n})$만큼의 시간도 길다고 판단해
이보다 좀더 효율적임을 위해서 결국 정확도를 조금 포기하고 확률적으로 소수인지 제대로 판단할 확률이 높은 알고리즘들이 나왔다.

\subsection{의사소수 판정(pseudoprime test)}

$n$이 소수일때 성립하는 페르마의 소정리를 판별방식으로 쓴다.

$n$과 서로소인 $a$에 대해서 $a^{n-1}$ 을 $n$으로 나눈 나머지는 무조건 1이 된다. 소수일때는 무조건 성립하니까 이를 판별 방식으로 쓰자는것

따라서 어떤 값 $n$에 대해서 $a^{n-1}$을 $n$으로 나눈 나머지가 1인지 판별해보면 된다.
적당한 $a$를 뽑고, $a^{n-1}$을
고속 지수승 알고리즘을 통해 $\log{n}$번에 구할수있다.

\begin{lstlisting}[style = CStyle]
PSEUDOPRIME(n)
    if MODULAR-EXPONENTIATION(2,n-1,n) != 1
        return COMPOSITE
    else return PRIME
\end{lstlisting}



이때 나오는 오진은 소수가 아닌데 $a^{n-1}$을 $n$으로 나눈 나머지가 1이 되는 경우이다.
이때 이 값을 \textbf{카마이클 수(Carmichael number)}라고 합니다 이 수의 특성도 재밌긴한데(사실 잘모름) 따로다루겠다.

카마이클수를 차례대로 나타내면

561, 1105, 1729, 2465, 2821, 6601, 8911, 10585, 15841, 29341, 41041, 46657, 52633, 62745, 63973

굉장히 띄엄띄엄 있어서 오진율이 낮다.




\subsection{밀러라빈소수 판별법(Miller-Rabin primality test)}

앞의 의사소수 판정을 조금더 보완하기 위해서 검사를 더 촘촘히 하기로 해보자

다음 증명된 사실을 이용한다.

\begin{theorem}
    홀수인 소수 $p$와 정수 $1 \le e$에 대해서 $\pmod{p^e} $에서 $x^2$을 n으로 나눈 나머지가 1이 되는 x의 해는 무조건 $1$,$-1(=n-1)$이다.
\end{theorem}

\begin{proof}
    $p^e \mid (x+1)(x-1)$이 될때 $p^e$는 $(x-1)$,$(x+1)$둘 중에 하나만이 될수가 있다. $p^e$가 $(x+1), (x-1)$둘다 나눌수있을때에는 $p^e$가 $2$로 나누어지기 때문이다. 만약 $p^e$가 $x+1$을 나눌수있는 경우 $x \equiv -1 \pmod{p^e}$, $p^e$가 $x-1$을 나눌수있는 경우 $x \equiv 1 \pmod{p^e}$ 
    가된다.
\end{proof}
이 정리의 대우에 따라서 1,-1을 근으로 가지지 않은 n은 합성수라고 판단 할 수 있다.

따라서 이 나머지가 1이되는데 x가 +-1인지를 살펴 보면된다.

임의의 a에 대해서 판단하는 방법은 다음과 같다.
\begin{enumerate}
    \item $n-1$ 를  $2^{t*u}$ (d는 홀수)로 나타낸다.
    \item $a^u$부터 $a^{2^t*u}$로 점점 제곱하면서 이 사이에 값이 1,n-1인데 그전의 값이 +-1이 아닌지 판별을 한다.
    \item 마지막에 $a^{2^t*u}$ 값이 1인지 비교한다.(페르마 소정리)
\end{enumerate}

예를 들어 카마이클 수는 561을 a = 2로 해서 구해보면
마지막에 $a^(2^t*d)$이 1이 되지만 제곱하기전의 값이 1이 아니라서 여기서 걸러지게 된다.

근데 이 판별이 결국 a에 따라 갈리게 된다. a값이 n에 대해서 제곱했을때 1이 되는 근이 아니어야 판별이 가능하다. 이는 합성수 n에 대한 다음 판별 방식으로 탐지되는 근이 $1 \sim n-1$사이에 최소 n/2가 존재한다

확실한건 a를 $2 \sim n/2$로 정하면 확실하게 나온다.


a를 촘촘하게 여러번 선택해서 판별하면 되는데 이러면 또 시간이 오래걸린다

따라서 탐지율이 a를 뽑는 횟수에 따라 다르다.

알고리즘 복잡도는 $a$를 반복해서 뽑는 $k$에 따라서 $O(k \log^3 n)$이다. 추가적으로 곱셈을 FFT로 처리했을때 $O(k \log^2 n)$까지 줄일 수 있다.

추가로 난제중 하나인 리만가설이 맞다면 $2\log^2 n$ 개의 a로 검사를 했을때 소수일것이라고 판단이 되었을 경우 확정적으로 소수임이 드러나기때문에 $O(\log^4 n)$의 시간복잡도를 가지는 소수판별법이된다.

\section{pollard's rho algorithms}

의사난수를 발생시켜 구하는 방법이다....


$ g(x)  = x ^2-1 \pmod{n} $

다음의 식을 사용해 반복적으로 $g(x)$를 만들어내고 
$y = g(g(x))$로 $\gcd(y-x, n)$이 $1,n$인지 검사하는 방식이다. 그뒤 $x = g(x)$로 업데이트하여 반복한다.


