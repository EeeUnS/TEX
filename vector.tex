\documentclass{oblivoir}
\usepackage{amsthm}
\usepackage{ikps}
\usepackage{thmtools}

\newtheorem{theorem}{Theorem}[section]
\newtheorem{corollary}{Corollary}[theorem]
\newtheorem{lemma}[theorem]{Lemma}

\declaretheoremstyle[% spaceabove=6pt,spacebelow=6pt, headfont=\color{MainColorOne}\sffamily\bfseries, notefont=\mdseries, notebraces={[}{]}, bodyfont=\normalfont,
headpunct={},
postheadspace=1em,
%qed=▣,
]{maintheorem}

\declaretheorem[%
name=정의,
style=maintheorem,
numberwithin=section, shaded={%bgcolor=MainColorThree!20,
margin=.5em}]{dfn}
% \begin{dfn}[]
% \end{dfn}

\newcommand{\onetok}[1]{ {#1}_1, {#1}_2, ... , {#1}_k}
\newcommand{\sumtok}[2]{  {#1}_1{#2}_1 + {#1}_2{#2}_2 + \cdots + {#1}_k{#2}_k}
\newcommand{\cvecthree}[3]{ \begin{pmatrix}    {#1} \\    {#2} \\    {#3} \\ \end{pmatrix}}
\begin{document}
    
여기서 대부분의 좌표표현은 벡터이고 행이아닌 열벡터로 표현합니다

$ (1, 1, 1)  =  \cvecthree{1}{1}{1} $
는 같습니다

\section{일차결합}
\begin{dfn}[일차결합]
    $\onetok{v}$를 n차원 벡터라 하자. 임의의 스칼라 $\onetok{a}$에 대하여
    벡터 $v = \sumtok{a}{v}$ 는 $R^n$에 속하는 한 벡터이다. 이것을 주어진 벡터 
    의 일차결합이라 한다.   
\end{dfn}

ex : 임의의 3차원 벡터 $x$는 $x = x_1i + x_2j + x_3k$와 같이 벡터 $ i = (1,0,0) , j = (0,1,0), k = (0,0,1)$의 일차결합으로 나타낼 수 있다.
\marginpar{$i,j,k$벡터(물리에서 주로사용)는 고등학교에서 기본 단위벡터로 $e_1, e_2, e_3$와 같습니다.}

\reversemarginpar
\marginpar{내분점 외분점은 모두 일차결합의 한 형태입니다.}





\section {span}

\begin{dfn}[span]
    
유클리드 벡터공간 $R^n$의 벡터 $\onetok{v}$에 대하여, 이 벡터의 일차결합 의 전체집합을 $Span(\onetok{v})$으로 나타내자. 즉,
\[
    Span(\onetok{v}) =\{ \sumtok{a}{v} \:|\: \onetok{a} \in R\}
\]

유클리드 벡터공간 $R^n$의 벡터 $\onetok{v}$에 대하여,

\[
    W = Span(\onetok{v})
\]
일 때, $\onetok{v}$은 $W$를 생성(span)한다고 하고, $W$를 $\onetok{v}$에 의하여 생성된 공간(span)이라고 한다 즉, $\onetok{v}$가 $W$ 를 생성한다는 것은  $W$가 $\onetok{v}$의 모든 일차결합을 포함하고 $W$의 임의의 벡터를 $\onetok{v}$의 일차결합으로 나타낼 수 있다는 것을 의미한다.
\end{dfn}

- 2 -
ex
\[
    Span(e_1, e_2, e_3) = R^3
\]
\[
    Span(\cvecthree{1}{0}{1},\cvecthree{-3}{1}{1},\cvecthree{-2}{1}{2}) = R^3
\]
 (추후에 더 다룹니다)

여기서 좌표는 시점인 원점과 종점인 한점을 잇는 벡터(=위치벡터)를 기본단위벡터 $e_1, e_2, e_3$의 일차결합으로 나타내고 계수만을 따온것이라는걸 알 수 있습니다.

즉 좌표 = 위치벡터 그자체로도 볼수있는셈
또한 우리가 왜 내분점 외분점을 이용해서 한 벡터를 여러 가지의 벡터로서 표현하는지를 알 수 있습니다






\section{부분집합}

정의:
다음 세 조건을 만족하는 R n의 부분집합 를 R n의 부분공간(subspace)이라고 한다.
1)  
2) 이면 이다.
3)  이면, 임의의 스칼라 에 대하여  이다.
조건 2)와 3)은 다음 하나의 조건 4)로 바꿀 수 있다.
4) x 이면, 임의의 스칼라 에 대하여 이다.
-----------------------------------------------------------------------------
보조정리
 은 R n의 부분공간이다
증명은 충분히 혼자 직접하실 수 있습니다.
- 3 -
일차독립과 일차종속
유클리드 벡터공간 R n의 벡터 를 생각하자 이 벡터에 의하여 생성된 부분공간을  라
하자.
벡터 가운데 어떤 한 벡터를 제외한 나머지 벡터들이 여전히 를 생성한다고 하자.
이것은 제외된 벡터 를 나머지 벡터들의 일차결합으로 나타낼 수 있음을 의미한다.
이와 같이, 가운데 어떤 한 벡터를 나머지 벡터들의 일차결합으로 나타낼 수 있을 때,
벡터 는 일차종속 (linearly dependent)이라고 한다.
단, 영벡터 0는 일차종속으로 정한다.
일차종속이 아닌 경우 즉, 벡터 가운데 어느 것도 나머지 벡터들의 일차결합으로 나타
낼 수 없을 때는 일차독립(linearly independent)이라고 한다.
가 일차종속일 필요충분조건은 적어도 하나는 0이 아닌 어떤 스칼라 에 대
하여 
가 일차독립일 필요충분조건은 임의의 스칼라에 대하여
 이면 이다.
ex)
1)

(같은 줄끼리 계산했다고 생각하시면됩니다)
눈에 잘보여서 이렇게 끝내지만 잘안보일때는
- 4 -

  인    의 값을 찾는겁니다
이를 푸는법은

    
      
인 연립방정식을 푸는것과 같습니다
2)  는 일차 독립이다
기저(basis)
R n의 부분공간의 벡터에 대하여
1)가 를 생성하고
2) 가 일차독립일 때
를 의 기저(basis)라고 한다.
-----------------------------------------------------------------------------
기저의 개수가 기저를 만드는 공간의 차원을 나타낸다.
ex)

는 둘 다 R의 기저이다.
 하나의 공간에 여러개의 기저가 정의될 수 있다.
이를이용하면 R 상에서 좌표계를 45도 꺽어서 새로운 좌표계를 정의 할수도있죠
이를 선형변환이라합니다
앞의 예제에 나온
은 결국에 뭘 나타낼까라는 의문이 들 수 있습니다
세 벡터는 일차종속이며 따라서

입니다
이 두 벡터는 일차독립이므로 이 공간상의 기저가 될 수 있고 이 공간이 2차원 평면이라는 정보까지는
알아냈습니다
평면의 결정조건이 서로다른 세점(=위치벡터) (0,0,0),(1,0,1),(-3,1,1) or
한점에서만나는 서로다른 두직선 벡터를 실수배하면 원점을 지나는 직선을 나타낸다 당연히 원점에서
만나겠죠
- 5 -
이 공간상의 한점을라 했을 때 이는 곧

라는 일차결합식으로 나타낼 수 있습니다
임의의 스칼라 에 대해서 생성된 공간의 모든 점을 나타낼 수 있습니다.
이를 풀어쓰면

어디서 많이 보던 모양아닌가요
직선의 방정식을 나타내는 여러 가지 방법중
벡터 방정식
방향벡터와 직선위의 한점일 때 직선위의 점 는

이것과 똑같은 생김새죠
가 평행이동을 나타내는 것이였다면 는 결국 직선의 기저를 나타내는 것입니다.
평면의 방정식을 구할 때도 직선의방정식을 구할때의 t를 소거하는 식과 마찬가지로
를 소거하여 구하면
 이라는 평면의 방정식이 나옵니다
 
\end{document}