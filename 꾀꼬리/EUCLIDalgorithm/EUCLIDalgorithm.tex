\documentclass{oblivoir}
    \usepackage{ikps,ansform}
    \usepackage{lipsum}
  
    \newcounter{problem}[section]
    \newenvironment{problem}{\noindent\refstepcounter{problem}\textbf{\large\theproblem.} }{}
    
    
    
\begin{document}
유클리드 호제법을 증명 해봅시다.
\par
\begin{justbox}
$d,m,n,$이 어떤 정수일 때 \par
1. $d$가 $m$과 $n$의 공약수일때, $m+n$도 $d$의 약수이다.
\par
2. $d$가 $m$과 $n$의 공약수일때, $m-n$도 $d$의 약수이다.
\end{justbox}
이에 대한 증명은 간단합니다. \par
$m=dq_1 , n = dq_2$( $q_1,q_2$는 어떤 정수)\par
$m+n = d(q_1 + q_2) , m-n=d(q_1-q_2)$ \par
\vspace{1\baselineskip}
참고로 $d$가 $n$의 약수(인수)일 때 $d\: |\: n$으로 표시합니다.
\par 
$m$과 $n$의 최대공약수는 $gcd(m,n)$이라고 합니다.\par
$r$이 $a$를 $b$로 나눈 나머지라면  $r=a\: mod \:b$입니다. \par
이를써서 위 명제를 다시 적으면 $d\:  |\:  n , d\:  | \: m $ $=>$ $d \: | \: (m+n), d\:  | \: (m-n)$
\par
\begin{justbox}
유클리드 호제법 \par
$a$가 음이 아닌 정수이고, $b$가 양의 정수이며, $r$이 $a$를 $b$로 나눈 나머지라면 $a$와 $b$의 최대공약수는 $b$와 $r$의 최대공약수와 같다.
\end{justbox}
위 명제를 위에 적었던 표기법을 사용하면 \par
$a$가 음이 아닌 정수이고, $b$가 양의 정수이며 $r=a\: mod\: b$이면 $gcd(a,b) = gcd(b,r)$이다.
\par
뭐 어쨋든, 증명을 하자면 $a=bq +r (0 \le r\: <\: b , q$ 는 어떤 정수 )인데, $c$를 $a$와  $b$의 공약수라 하면, $c$는 $bq$의 약수인 것은 자명합니다. $a$또환 $c$의 약수이므로 $c$는 $a-bq\:(=r)$의 약수입니다. 따라서 $c$는 $b$와 $r$의 공약수입니다. 반대로 $c'$가 $b$와 $r$의 공약수이면, $c'$는 $bq+r(=a)$의 약수가 되고 따라서 $a$와 $b$의 공약수가 됩니다. 따라서 $a$와 $b$의 공약수 집합이 $b$와 $r$의 공약수 집합과 같으므로 $gcd(a,b) = gcd(b,r)$이 성립합니다.\par

\vspace{1\baselineskip}
유클리드 호제법을 이용해 $b,r$를 새로운 $a,b$로서 보고 연속해서 사용하면,  $a$자리에는 최대공약수가 $b$는 0이 됩니다.
\end{document}