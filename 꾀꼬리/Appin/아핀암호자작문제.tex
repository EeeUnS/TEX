\documentclass{oblivoir}
    \usepackage{ikps,ansform}
    \usepackage{lipsum}
    \usepackage{titlesec}
    \newcounter{problem}[section]
    \newenvironment{problem}{\noindent\refstepcounter{problem}\textbf{\large\theproblem.} }{}
    \setcounter{tocdepth}{4}
      \setcounter{secnumdepth}{3}
\begin{document}
\section{}
입력파일  \href{https://www.dropbox.com/s/wd20ebjvi4m6m0c/aligned.txt?dl=0}{aligned.txt}(2-3주차 과제)를 읽는다. 이 파일을 키값 (7,2)인 아핀암호기법을 사용하여 \href{https://www.dropbox.com/s/9d30fxbhz2nr869/encrypt.txt?dl=0}{encrypt.txt}에 출력하고 이 암호화 한것을 다시 복호화하여  \href{https://www.dropbox.com/s/vm6ims2kz18nfez/decrypt.txt?dl=0}{decrypt.txt}에 출력하는 프로그램을 작성하여라 해당 txt 파일은 키값이 7인 덧셈암호를 이용하여 나타낸것이다.
복호화시에는 모듈러 연산에대한 역수를 곱하기위해 확장된 유클리드 알고리즘을 사용해야한다.
\par
아핀암호 참고: \url{http://egloos.zum.com/eyestorys/v/3544631}

\end{document}