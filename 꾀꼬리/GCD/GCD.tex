\documentclass{oblivoir}
    \usepackage{thmtools}

\usepackage{amsthm}
 
    \newtheorem{theorem}{Theorem}[section]
    \newtheorem{corollary}{Corollary}[section]
    \newtheorem{definition}[section]{Definition}

\begin{document}
\begin{theorem}
$f(n) = 1+10+\cdots +10^n$이라 하자.

$ \gcd(f(x) , f(y)) = f(\gcd(x,y)) $임을 보여라.
\end{theorem}
%...증명
\begin{corollary}
    $\alpha > \beta$일때, 다음이 성립한다.
    
    $\gcd(\alpha ,\beta) = \gcd(\alpha-\beta , \beta)$  
\end{corollary}
\begin{proof}
    $\alpha$, $\beta$의 최대 공약수를  $x$라 하자.
    $\alpha = x \cdot a , \beta = x \cdot b$ ($a,b$는 $a>$b이며 서로소인 두 정수)이며, $\alpha -\beta = x(a-b)$이다 $a-b$는 $b$와 서로소이며 두 값의 최대공약수는 여전히 $x$이다.
\end{proof}

$\gcd(f(x)-f(y),f(y)) = \gcd(f(x),f(y)) = \gcd(f(x-y) \cdot 10^{y},f(y)) $ 

이때 $10^{y}$와 $f(y)$는 항상 서로소이므로 $\gcd(f(x-y), f(y))$가 성립합니다. %...증명

따라서 유클리드 호제법을 전개했을때, $\gcd(f(x), f(y)) = \gcd(f(\gcd(x,y)),0)$이 되고 
이는$f(\gcd(x,y))$과 같습니다.

\end{document}