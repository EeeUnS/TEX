\documentclass{oblivoir}
\usepackage{graphicx}
\usepackage{ikps,ansform}
  \newcounter{problem}[section]
    \newenvironment{problem}{\noindent\refstepcounter{problem}\textbf{\large\theproblem.} }{}
   
\begin{document}
   
$x>a$에서 정의된 함수 $f(x)$와 최고차항의 계수가 $-1$인 사차함수 $g(x)$가 다음 조건을 만족시킨다. (단, $a$는 상수이다.)
   \begin{condition}{(가)}
   \item $x>a$인 모든 실수 $x$에 대하여 $(x-a)f(x)=g(x)$이다.
   \item 서로 다른 두 실수 $\alpha, \beta$에 대하여 함수 $f(x)$는  $x=\alpha$와 $x=\beta$에서 동일한 극댓값 $M$을 갖는다. (단, $M>0$)
   \item 함수 $f(x)$가 극대 또는 극소가 되는 $x$의 개수는 함수 $g(x)$가 극대 또는 극소가 되는 $x$의 개수보다 많다.
   \end{condition}
 

$\beta - \alpha = 6\sqrt{3}$일 때, $M$의 최솟값을 구하시오. [ 4점 ]


-------------------------


식으로서 정리를 해봅시다


$f(x)는 x>a $에서 정의됨


$g(x) : -x^4$ ...


   \begin{condition}{(가)}
   \item $x>a$인 $\forall x$\,에대해 $(x-a)f(x)=g(x)$만족
   \item $f(\alpha)=f(\beta)=M$ 여기서 극댓값을 가지며 $f'(\alpha)=f'(\beta)=0$ , $M > 0$
   \item $n(F) > n(G) \;(F = \{ x\,|\,f'(x)=0 \} , G = \{ x\,|\,g'(x) =0 \}) $
   \end{condition}
구해야하는 값이 $M$인데도 불구하고 도통 감이 잡히지 않습니다. 이 문제는 어디서부터 접근을 해야할지도 난감하죠.\par 일단 나에서 주어진 조건인
$f(x)$를 먼저 생각해봅시다. $f(x) =\dfrac{g(x)}{x-a}$ \par 첫번째 포인트는  $f(x)$가 삼차함수일거라는 고정관념을 벗어나는것에서 부터 시작합니다. 아직 잘 감이 안잡히니 일단 단순하게 나조건을 가조건에 대입을 해봅시다.\par
$f(x) =\dfrac{g(x)}{x-a}$ \par $f(\alpha)=f(\beta)=M=\dfrac{g(\alpha)}{\alpha-a}=\dfrac{g(\beta)}{\beta-a}$\par $f'(x) =\dfrac{-g(x)+(x-a)g'(x)}{(x-a)^2}$\par $f'(\alpha)=f'(\beta)=0=\dfrac{-g(\alpha)+(\alpha-a)g'(\alpha)}{(\alpha-a)^2}=\dfrac{-g(\beta)+(\beta-a)g'(x)}{(\beta-a)^2}$\par
음... 아직 잘 모르겠네요 남은 조건들을 좀더 분석해 봅시다. \par 일단 $f'(x)$의 분자를 봤을때 최고차항은 $-3x^4$이므로 최대 네개의 실근을 가질수 있습니다. 따라서 $n(F) <= 4$ 이네요 \par
또한 극댓값을 해석을 해보면 $x>a$에서 정의된 미분가능한 함수 $f(x)$에 대하여 사잇값 정리에 의해서 $f(x)$가 극솟값을 가지게 되는 $\theta$가 $( \alpha, \beta) $사이에 반드시 존재하게 됩니다.\par 따라서 $1<= n(G) <= 3<= n(F) <= 4$가 성립합니다. (단, $n(G) < n(F)$)\par 또한 이로얻은 식을 적어보면 \par
$f'(\theta) = 0 = \dfrac{-g(\theta)+(\theta-a)g'(\theta)}{(\theta-a)^2}$이 됩니다.\par
아직도 구해야하는 값에대한 단서도 결국 $f(x)$를 가지고 어쩌란건지도 잘 모르겠습니다. 정보는 가장 많이 줬지만 건들기가 까다로운 까닭에 해당 모든 조건들을 4차함수라고 주어진  $g(x)$에 대한 조건으로 바꿔서 $g(x)$를 일단 추론해보자라는 생각으로 수학(나)형의 고전적인 킬러의 접근 방식으로서 접근을 해봅시다\par
$g'(x) = f(x) + (x-a)f'(x)$ (단, $x>a$일 때) \par
$g'(\alpha) = f(\alpha) + (\alpha-a)f'(\alpha) =g'(\beta) = f(\beta) + (\beta-a)f'(\beta) = M $\par
$ g'(\alpha) = \dfrac{g(\alpha)}{\alpha-a}= g'(\beta)=\dfrac{g(\beta)}{\beta-a}=M $\par
자 이것을 해석해 봅시다. $(a,0)$과 $(\alpha,g(\alpha)$를 이은 평균변화율, $g(\alpha)$의 미분계수, $g'(\beta)$ ,$(a,0)$과 $(\beta,g(\beta))$의 평균변화율이 모두 $M$으로 같다는 겁니다. 여기서 $g'(\alpha)$에서의 접선을 생각할수 있는데 이것이 $(a,0)$과 $(\beta,g(\beta))$에 접하고 기울기가 M인 직선으로 볼수있습니다.\par\par
여기서 고등학교 1학년 지식을 사용하자면 일차함수는 지나는 한점과 기울기만 주어지면 확정적으로 함수가 정해지는데 이에 의해서 $( \alpha ,g( \alpha )) ,(\beta,g(\beta))$가 한 직선위에 있음을 알수있습니다. \par
이제 수학(나)형에서 자주썼던 테크닉을 가져옵시다.
한 직선위의 두점에서 4차함수 g(x)가 접하고 있습니다. 따라서\par $g(x)-M(x-a) = -(x-\alpha)^2(x-\beta)^2$이 됩니다. 
\par$g(x)= -(x-\alpha)^2(x-\beta)^2+M(x-a)$가 됩니다 \par
드디어 구해야할 $M$이 의미있는 식으로서 등장을 했습니다. \par
유일한 부등식 조건인 (다)의 조건을 사용하면 M의 최댓값을 구할수있을 것으로 예상이 됩니다. 부등식을 다시 들고옵시다.$1<= n(G) <= 3<= n(F) <= 4$(단, $n(G) < n(F)$) 얻은 $g(x)$를 이용해서 다음 부등식에 어떻게 적용을 할까 싶은가 생각해보니 $g'(x)$의 근의 개수로서 극값의 개수를 분류를 하게된다면 부등식을 이용할 수있다는 생각을 하게됩니다. 따라서
$g'(x)$를 구합니다. \par
$g'(x)=-4(x-\alpha)(x-\beta)(x-\dfrac{\alpha+\beta}{2})+M$\par
역시나 $M$의 값에 따라서 $g'(x)$의 근의 개수가 나눠지는것을 알수있었습니다.\par
사실 잘생각해보면 $g(x)$의 극값은 1 또는 3 입니다. $n(G)=3$일때 M의 최솟값은 없습니다 $M>0$ 때문이죠 따라서 극값은 1개 일 수 밖에 없습니다. \par
이런 추정을 통해서 결과적으로 $n(F)=3$이 확정적으로 나온다는 것을 알수있습니다\par그럼 직접 구해봅시다.\par
이를 구할 생각나는 방법은 두가지인데 구해둔 $f'(x)$에 구한 $g(x),g'(x)$를 대입해서 근을 따져보는 방법이 있습니다. 가장 확실하고 계산이 복잡합니다. 제발 직접 한번 해보세요...\par\par
두번째는 추정으로부터의 확신인데 $f(x)$의 나머지 한근이 $x<=a$에 존재하는것을 밝히는 겁니다.\par
$f'(x) =\dfrac{-g(x)+(x-a)g'(x)}{(x-a)^2}$을 봅시다 $x=a$바로 왼쪽근방에서의 부호를 조사하면  $-g(x)$ , $(x-a)g'(x)$ , $(x-a)^2$ 모두 양수가 되어 전체 값이 양수가 됩니다 이제$\lim_{x \to -\infty}f'(x)$의 부호를 살펴보면 $-g(x)+(x-a)g'(x)$는 최고차항이 $-3x^4$이기에 음수 $(x-a)^2$는 양수로서 $\lim_{x \to -\infty}f'(x) < 0$이기에 따라서 사잇값 정리에의해 $(-\infty, a)$에서 적어도 근 하나를 가집니다 당연히 $f(x)$의 정의역 밖이므로 $f'(x)$의 근으로서 채택될수 없습니다. 따라서 $n(F)=3$입니다.\par
마지막으로 계산만 남았군요. $\alpha = -3\root{3} , \beta = 3\root{3}$으로 놓고 풀어도 문제가 없습니다. $\alpha$는 구해지지 않는 값이고 마지막 계산과정에서 결국에 소거되는 문자기에 어떻게 놓아도 상관이 없죠 이 문제를 시작으로 뒤에나온 평가원 시험에도 이런 류의 문제가 출제 된 적이 있습니다.\par
$M$이 최소가되려면 $g'(x)$는 $M>0$인 부분에서 한개의 극값을 가지는 실근 1개 또는 이중근과 다른 실근 하나를 가지면 됩니다. 이는 $g'(-3)=0$을 계산 하면됩니다

\end{document}