\documentclass{oblivoir}
\usepackage{amsthm}
\usepackage{thmtools}

\declaretheoremstyle[% spaceabove=6pt,spacebelow=6pt, headfont=\color{MainColorOne}\sffamily\bfseries, notefont=\mdseries, notebraces={[}{]}, bodyfont=\normalfont,
headpunct={},
postheadspace=1em,
%qed=▣,
]{maintheorem}

\declaretheorem[%
name=정의,
style=maintheorem,
numberwithin=section, shaded={%bgcolor=MainColorThree!20,
margin=.5em}]{thm}

\begin{document}
\begin{thm}[isomophic] 두 그래프 G와 H가  전단사 함수 $\theta : V(G) \longrightarrow V(H)$와 $\phi : E(G) \rightarrow E(H)$에 대해 다음이 성립하면 두 그래프는 동형(isomophic)이다.
\begin{itemize}
    \item $\psi(e) = us( e \in E(G), u,s \in V(G)) $
    \item $\psi(\phi(e)) = \theta(u)\theta(v)$% $\phi(e)$ %($\phi(e) \in E(H)$)
\end{itemize}
\end{thm}
1.2.5

명제: $G \cong H$ ,둘다 simple graph 일때, $$ uv \in E(G) \Longleftrightarrow  \theta(u)\theta(v) \in E(H)$$

정의로 부터 $\psi(e) = us$인 간선 $e$ ($e \in E(G)$)에 대해 대응되는 $\psi(\phi(e)) = \theta(u)\theta(v)$인 $\phi(e)$($\phi(e) \in E(H)$)가 존재함을 알 수 있다. 따라서 $ uv \in E(G) \rightarrow  \theta(u)\theta(v) \in E(H)$ 성립, 반대의 경우도 마찬가지로 성립한다.

\end{document}
