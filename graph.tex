\documentclass{oblivoir}

\begin{document}
1.2.5
$G \cong H$ , simple

bijection $\theta : V(G) \longrightarrow V(H)$
$ uv \in E(G) \Leftrightarrow  \theta(u)\theta(v) \in E(H)$

정의로 부터 $\psi(e) = us$인 간선 $e$ ($e \in E(G)$)에 대해 대응되는 $\psi(\phi(e)) = \theta(u)\theta(v)$인 $\phi(e)$($\phi(e) \in E(H)$)가 존재함을 알 수 있다. 따라서 $ uv \in E(G) \rightarrow  \theta(u)\theta(v) \in E(H)$ 성립, 반대의 경우도 마찮가지로 성립한다.

\end{document}
