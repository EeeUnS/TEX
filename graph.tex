\documentclass{oblivoir}
\usepackage{amsthm}
\usepackage{thmtools}

\declaretheoremstyle[% spaceabove=6pt,spacebelow=6pt, headfont=\color{MainColorOne}\sffamily\bfseries, notefont=\mdseries, notebraces={[}{]}, bodyfont=\normalfont,
headpunct={},
postheadspace=1em,
%qed=▣,
]{maintheorem}

\declaretheorem[%
name=정의,
style=maintheorem,
numberwithin=section, shaded={%bgcolor=MainColorThree!20,
margin=.5em}]{thm}

\begin{document}
\begin{thm}[isomophic] 두 그래프 G와 H가  전단사 함수 $\theta : V(G) \longrightarrow V(H)$와 $\phi : E(G) \rightarrow E(H)$에 대해 다음이 성립하면 두 그래프는 동형(isomophic)이다.
\begin{itemize}
    \item $\psi(e) = us( e \in E(G), u,s \in V(G)) $
    \item $\psi(\phi(e)) = \theta(u)\theta(v)$% $\phi(e)$ %($\phi(e) \in E(H)$)
\end{itemize}
\end{thm}

  $H \subseteq G$ if $ V(H) \subset V(G), E(V) \subset E(G)$, and $\psi_{H}$ is restricton $\psi_{G}$
    subgraph : H, supergraph: G

    $ H \subset G$ proper graph: H
    $V(H) = V(G) , H \subseteq G \Longleftrightarrow$ spaning subgraph : $H$

    spaning subgraph  simple graph $\rightarrow$ undelying simple graph


\begin{itemize}
    \item 1.2.5
    $G \cong H$ , simple

    bijection $\theta : V(G) \longrightarrow V(H)$
    $ uv \in E(G) \Leftrightarrow  \theta(u)\theta(v) \in E(H)$

    정의로 부터 $\psi(e) = us$인 간선 $e$ ($e \in E(G)$)에 대해 대응되는 $\psi(\phi(e)) = \theta(u)\theta(v)$인 $\phi(e)(\phi(e) \in E(H))$가 존재함을 알 수 있다. 따라서 $ uv \in E(G) \rightarrow \theta(u)\theta(v) \in E(H)$ 성립, 반대의 경우도 마찬가지로 성립한다.
    \item 1.2.11
    \begin{itemize}
        \item 여 그래프(complement graph) : 모든 정점에 대해서 포함하고 있는 존재하는 간선은 제거, 존재하지않는 간선을 생성해서 만든 그래프
        \item 자기 여 그래프 (self-complementary graph) : 여그래프와 자기자신이 동형인 그래프
    \end{itemize}
    (b): 자기 여 그래프가 되기위해선 일단 동형 이전에 간선의 갯수가 동일해야하는데 여기서 총 생길수있는 간선의 갯수는$\dfrac{vX(v-1)}{2}$ 가 최댓값이자 그래프의 간선수 + 여그래프의 간선수 입니다. 그래프의 간선수 = 여그래프의 간선수 이므로 $v$나 $v-1$은 적어도 둘 중 하나는(적어도지만 사실 둘다 4의 배수인 경우의 수는 존재하지않습니다) 4의 배수여야합니다
    따라서 $v\pmod{4}$는 0 또는 1
    
    \item 1.5.2

    M'는 M의 전치행렬 원표기 $M^{T}$,
    $MM'[v_i][v_i] = \sum_{j=1}^n M[v_i][e_j] \cdot M'[e_j][v_i]$

    $ M'[e_j][v_i] = M[v_i][e_j] $이며 simple graph일때 각 값은 0 또는 1이기 때문에 결과적으로 대각선의 값은 해당 정점의 차수가 된다.

    $A$ 행렬에서 $A[v_i][v_j] = A[v_j][v_i]$ 

    $d(i) = \sum_{j=1}^n A[v_i][v_j] = \sum_{j=1}^n A[v_j][v_i]$이다.
    마찬가지로 simple graph에서 $A[v_i][v_j]$은 무조건 0 또는 1을 가지므로 $A[v_i][v_j] \cdot A[v_j][v_i] = A[v_i][v_j]$이다.

    $A^2$에서 $A^2[v_i][v_i] = \sum_{j=1}^n A[v_i][v_j] \cdot A[v_j][v_i]= \sum_{j=1}^n A[v_i][v_j] = d(i)$

    k-regular graph : 정규그래프
    집합 $A$의 $|A|$ : A의 원소의 갯수

    \item 1.5.3
    k-regular bipartite graph의 bipartition($X$,$Y$)이 $|X|\neq |Y|$라 하자. $d(v)=|Y|,\: d(u)=|X|(u \in Y )$ $d(v) \neq d(u)$ 이는 k-regular graph의 조건에 모순

    \item 1.5.4

    두명 이상의 사람이 있는 그룹에서 그룹 내 친구의 수(그룹 내부의 사람으로 제한)가 같은 사람이 반드시 두명이 있음을 보여라

    사람이 n명일때 친구의 수는 최대 n-1명이기때문에 비둘기집의 원리에 의해 친구 수가 같은 사람이 무조건 두명이 존재한다.

    각각의 사람을 정점, 친구관계를 간선으로 나타낸다면은 해당 그룹은 simple graph로 볼수있으며 친구의 수는 각 정점의 차수가 된다.

    따라서 해당 문제는 simple graph일때 반드시 두 정점의 차수가 같음을 보이는 것과 같다. 
    \item 1.5.5
    ??
    \item 1.5.6 : A sequence $d = (d_1, d_2 , \cdots , d_n)$ is graphic if there is a simple graph with degree sequence d. Show that

    (a) (7,6,5,4,3,3,2) : 정점이 총 7갠데 첫번째 정점의 간선이 7개인것은 simple graph의 조건을 충족하지 못한다.
    (6,6,5,4,3,3,1) : 총 7개의 정점중 simple graph라면 자신을 제외한 모든 정점에 간선을 잇는 차수가 6인 정점이 2개이지만 차수가 1인 정점이 있으므로 모순이다.

    (b) if $d$ is graphic and $d_1 \le d_2 \le ... \le d_n$, then $\sum_{i=1}^{n} d_i$ is even and $\sum_{i=l}^{k} d_i \le k(k -1)+\sum_{i=k+1}^{n}\min(k, d_i)$ for $1 \le k \le n$


    \item 1.6.8

    (a)간선 e가 빠짐으로서 하나였던 component가 두개의 component가 될 요지가 있다. 따라서 $\omega(G) \le \omega(G-e) \le \omega(G)+1 $가 성립한다.

    (b) inequality: 부등식
    반례: $V(G) = { v_1, v_2, v_3} ,\: E(G) = { e_1 , e_2} ,\: \psi_H(e_1) = v_1v_1 ,\psi_H(e_2) = v_2v_3  $
    $v_1$과 $v_2v_3$가 각각 연결되어있는 $\omega(G) = 2$인 그래프이다$v_1$을 제거할때 component가 하나 사라지므로 주어진 부등식을 만족하지 못한다.

    \item 1.6.14
    $uv, uw, uw \in E $이면 $G$는 complete가 되기때문에 $uw \notin E$


    \item 1.7.2
    simple graph가 아닌경우
          루프를 포함하는경우 $v_0v_0$는 정의에 의해 사이클이다.
               임의의 정점$v_0, v_1$에 간선이 2개이상인경우            $v_0v_1v_0$사이클을 이룬다
         
        simple graph인경우
            정점의 개수가 k인 그래프를 생각하자. 이때 $v_0v_1v_2 \cdots v_i$인 서로 다른 정점만 최대한 이어진 연결을 생각해볼때 i의 차수는 명제의 조건에의해서 무조건 $0<j \le k$인 정점 $v_j$에 연결이 되어있어야한다. 따라서 $v_{j}v_{j+1} \cdots v_k$인 사이클을 이룬다.

    \item 1.8.4
    늑대, 염소, 양배추가 강가에 있습니다.

페리맨은 그들을 데려 가기를 원하지만 보트가 작기 때문에 한 번에 하나만 가져갈 수 있습니다.

 명백한 이유 때문에, 늑대와 염소, 염소와 양배추는 보호받지 못하게 될 수 없습니다.
 
도선장은 어떻게 강을 건너 갈거야?

그래프랑 무슨 관련이 있는진 모르겠지만
먼저 양을 옮긴다. 그다음 늑대를 옮긴다. 돌아가면서 양을 다시 데리고간다. 양을 두고 양배추를 옮긴다. 양을 옮긴다.

    \item 1.8.5
    \item 1.8.6

\end{itemize}
\end{document}
