\documentclass{oblivoir}
    \usepackage{ikps,ansform}
    \usepackage{lipsum}
\usepackage[utf8x]{inputenc}
    \newcounter{problem}[section]
    \newenvironment{problem}{\noindent\refstepcounter{problem}\textbf{\large\theproblem.} }{}
    
    
    
\begin{document}

모집단 : 대한민국 성인 남성 (편의상 2천만명으로 침)\par
모평균 : 진짜 평균. 전수조사 안하면 모르는 값. $m$이라고 씀.\par
모표준편차 : 진짜 표준편차. 전수조사 안하면 모르는 값. $\sigma$라고 씀.\par
\par
표본평균 : n명 뽑아서 걔들로 구한 평균. $\bar{X}$라고 씀.\par
표본표준편차 : n명 뽑아서 걔들로 구한 표준편차. $\mathrm{s}$라고 씀.\par
\par\par

\section{}
표본평균  $\bar{X}$는 일종의 확률변수이다\par
n명 임의추출해서 구할 때마다 표본평균값이 확률적으로 정해지니까 $\bar{X}$는 연속확률변수이다.\par
우리가 증명할 필요는 없고 수학자들이 알아낸 바에 따르면\par
표본평균은 (어차피 문제풀땐 상관없이 대체적으로) 기댓값이 $m$, 표준편차가 $\dfrac{\sigma}{\sqrt{n}}$인 정규분포를 따른다.\par
여기서 조심할 건, '표본평균의 표준편차'는 그냥$\dfrac{\sigma}{\sqrt{n}}$일 뿐이고, '표본표준편차'인 $\mathrm{s}$와는 하등 무관함.\par
\begin{center}
    
\textbf{하등 무관함. 하등 무관함. 하등 무관함. 하등 무관함. 하등 무관함. 하등 무관함. 하등 무관함. 하등 무관함. 하등 무관함. 하등 무관함. 하등 무관함. 하등 무관함. 하등 무관함. 하등 무관함. 하등 무관함. 하등 무관함. 하등 무관함. 하등 무관함. 하등 무관함. 하등 무관함. 하등 무관함. 하등 무관함. 하등 무관함. 하등 무관함. 하등 무관함. 하등 무관함. 하등 무관함. 하등 무관함. 하등 무관함. 하등 무관함. 하등 무관함. 하등 무관함. 하등 무관함. 하등 무관함. 하등 무관함. 하등 무관함. 하등 무관함. 하등 무관함. }

\end{center}
\section{}
조사한 표본평균 하나로 모평균이 어떤 범위에 들어올 확률을 개략적으로 구하는 방법\par
표본평균이 정규분포를 따른다는 점을 이용하여\par
정규분포 확률 구하는 방식을 역이용하여 적절히 수식을 변형하면\par

$\bar{X} - a\dfrac{\sigma}{\sqrt{n}}\le m \le \bar{X} + a\dfrac{\sigma}{\sqrt{n}}$
\par
라는 부등식이 성립할 확률이 '적당한 상수'에 의해 정해진다는 사실을 알 수 있다.\par
($a$ = $1.96$이면 $95\%$ , $2.58$이면 $99\%$)
\par\par

\section{}
2번은 탁상공론. 인생은 실전이다. 대충 되면 그만이야.\par
2번은 참이긴 하겠지만 현실에 적용할 수 없다. 시그마는 아까 말했듯이 '전수조사 안하면 모르는 값'이니까.\par
그래서 아예 모르는 값인 시그마 대신에, 지금 알고 있는 값인 S를 대체하여 써도 허용됨. 안그럼 추정 자체를 못하니까.\par
\par
따라서 현실적으로 우리가 세우는 부등식은\par
\par
$\bar{X} - a\dfrac{\sigma}{\sqrt{n}}\le m \le \bar{X} + a\dfrac{\sigma}{\sqrt{n}}$\par
인 척 하는\par
$\bar{X} - a\dfrac{\mathrm{s}}{\sqrt{n}}\le m \le \bar{X} + a\dfrac{\mathrm{s}}{\sqrt{n}}$\par
이고, 이걸로 풀면 됨.\par
\par
끝.


\end{document}