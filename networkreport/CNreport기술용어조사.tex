\documentclass{oblivoir}
\usepackage{graphicx}
\usepackage{hyperref}
\usepackage{ikps,ansform}
  \newcounter{problem}[section]
    \newenvironment{problem}{\noindent\refstepcounter{problem}\textbf{\large\theproblem.} }{}
         \setcounter{secnumdepth}{3}

\begin{document}
\title{기술용어조사 레포트}
\author{이윤승 201712052}
\maketitle
\section{DoS(Denial of Service attack)}
시스템을 악의적으로 공격해 해당 시스템의 자원을 부족하게 하여 원래 의도된 용도로 사용하 지 못하게 하는 공격의 통칭
\subsection{type}
\subsubsection{DDoS(Distributed DoS)(분산 서비스 거부 공격)}
DoS의 큰규모로서 대규모의 DoS attack을 말한다
\subsubsection{Application layer attacks}
OSI 모델의 어플리케이션 계층을 대상으로 하는 DDoS공격 방식이다.

\subsubsection{Advanced persistent DoS (APDoS)}큰 네트워크 계층에 DDoS공격과 어플리케이션 계층(HTTP)에 집중하며 SQLI\footnote{SQLI(SQL injection) : code injection의 일종으로 의도하지 않은 SQL문이 실행되게 함으로써 데이터베이스를 조작하는 방법이다.(code injection: 코드를 악의적으로 삽입해서 타겟을 망치는 기법)}와  XSS\footnote{XSS(Cross-site Scripting) : 사용자가 사이트에 스크립트를 삽입하는 기법}공격이 다양하게 발생한다.

\subsection{증상}
네트워크 성능저하\newline
특정 웹사이트 사용불가\newline
모든 웹사이트 사용불가\newline
스팸이메일 폭탄

\section{virus}
스스로를 복제하여 컴퓨터를 감염시키는 컴퓨터 프로그램이다. 네트워크 또는 이동식 매체를 통해서 다른 컴퓨터를 통해 확산할 수 있다. 네트워크에 영향을 주지않으며 대상 컴퓨터에서만 활동을 한다.

\section{worm}
스스로 복제가능한 컴퓨터이며 독자적으로 실행되며 네트워크를 사용하여 자신의 복사본을 전송할 수 있다. 

\section{랜섬웨어(Ransomware)}
컴퓨터 시스템을 감염시켜 접근을 제한하고 일종의 돈을 요구하는 악성 소프트웨어의 한 종류이다. 암호화 알고리즘을 사용하여 파일데이터를 암호화한다.

\subsection{주요 감염경로}
신뢰할 수 없는 사이트\newline
스팸메일\newline SNS서비스\newline광고배너\newline P2P 사이트\newline어도비 플래시 등 경로가 매우 많다.

\subsection{작동원리}
\subsubsection{공격 대상 파일검색}
\subsubsection{파일 암호화}
고정키 암호화 \newline다이나믹키 암호화(암호화키 생성 후 암호화키 서버 전달)
\subsubsection{파일 이동}
\subsubsection{감염 안내 및 복구 방법 메세지 출력}

\subsection{사례}
Reveton\newline 크립토락커\newline CryptoLocker.F, TorrentLocker\newline CryptoWall\newline Fusob
워너크라이

\section{botnet(봇넷)}
인터넷에 연결되어있으면서 bot\footnote{bot(internet bot) : 자동으로 일을 수행하는 소프트웨어(or 스크립트)}에 감염된 장치를 말한다.
불법으로 감염된 봇의 경우 DDoS공격, 데이터조작, 스팸메일 등에 노출될 수 있으며 컴퓨터의 실질적인 컨트롤 권한이 제3자에게 넘어가있다.  대표적인 예시로 좀비PC가 있으며 이 컴퓨터들을 이용해 DDoS공격에 사용되기도 한다.
\section{Trojan horse(트로이목마)}
겉보기엔 정상적인 프로그램 같지만 실행시에 악성 코드를 실행한다. 이름의 유래 또한 그 때문 virus나 worm과 달리 다른파일에 삽입되거나 스스로 전파되지않는다. 최근에는 백도어 목적으로 많이 사용된다.
\subsection{예시}

\subsubsection{넷버스 (Netbus)}12345번 포트 사용. 가장 사용하기 쉽고 퍼지기 쉬운 트로이 중의 하나.
\subsubsection{스쿨버스 (Schoolbus)}
54321번 포트 사용.
\subsubsection{Executor}
80번 포트 사용. 감염된 컴퓨터의 시스템 파일을 삭제/시스템을 파괴하는 트로이 중의 하나.
\subsubsection{Silencer}
1001번 포트 사용. 제거 툴은 나와 있지 않음.
\subsubsection{Striker}
2565번 포트 사용. 감염된 컴퓨터를 무조건 고물로 만들어 버림. 이는 시스템 드라이브 등 하드디스크를 모두 파괴하여 아예 부팅이 안 되게 하는 트로이이기 때문.


\section{IOT(Internet of Things,사물인터넷)}
사물에 인터넷을 연결하는 기술 \newline 사용되는분야는 스마트홈, ,의료,헬스케어,수송,,농업, 제조 등등 활용분야가 매우 다양하다
네트워크란 점 때문에 보안에대한 우려가 제기된다. 실제로 2016년 DDoS공격을 통해 수십만대의 IoT디바이스가 감염되었다.

\section{클라우드 컴퓨팅(cloud computing)}
정보를 자신의 컴퓨터가 아닌 인터넷에연결된 다른 컴퓨터로 처리하는 기술
\subsection{장점}
초기 구입 비용과 비용 지출이 적으며 휴대성이 높다.\nuwline 컴퓨터 가용율이 높다. 이러한 높은 가용율은 그린 IT 전략과도 일치한다.\nuwline 다양한 기기를 단말기로 사용하는 것이 가능하며 서비스를 통한 일관성 있는 사용자 환경을 구현할 수 있다.\nuwline 사용자의 데이터를 신뢰성 높은 서버에 보관함으로써 안전하게 보관할 수 있다.\nuwline 전문적인 하드웨어에 대한 지식 없이 쉽게 사용 가능하다.

\subsection{단점}
서버가 공격 당하면 개인정보가 유출될 수 있다.\nuwline 재해에 서버의 데이터가 손상되면, 미리 백업하지 않은 정보는 되살리지 못하는 경우도 있다.\nuwline 사용자가 원하는 애플리케이션을 설치하는 데에 제약이 심하거나 새로운 애플리케이션을 지원하지 않는다.\nuwline 통신환경이 열악하면 서비스 받기 힘들다.\nuwline 개별 정보가 물리적으로 어디에 위치하고 있는지 파악할 수 없다.

\subsection{대표적인 모델(실제로 종류가 다양하다)}
\subsubsection{폐쇄형 클라우드}
오직하나의 단체를 위해서만 운영되는 클라우드
\subsubsection{공개형 클라우드}
공개적 이용을 위해 열린 네트워크를 통해 렌더링되는 클라우드 일반적으로 사람들이 사용하는 클라우드가 여기에 해당한다.
\subsubsection{혼합형 클라우드}
둘 이상의 클라우드 조합. 여러 개의 배치모델들이 있다.

\section{5G 이동통신(5G=fifth-generation)}
아직 채용되지않은 무선 네트워크 기술이다. 일반적으로 얘기하는 이동통신 표준 2g 3g 4g(LTE)의 다음 세대이다
\subsection{주요 기술}
\subsubsection{라디오 주파수를 새로 사용}
\subsubsection{작은셀(Small Cell)}
저전력 셀룰러 무선 엑세스 노드
\subsubsection{빔포밍(Beamforming)}
지향성 신호 전송와 수신을 위해 센서 어레이 에 사용되는 신호 처리 기술
\subsubsection{NOMA (Non-Orthogonal Multiple Access)}
다중 접속 기술
\subsubsection{대규모 MIMO 안테나}
다중입력 , 출력 안테나
\newpage
\section{출처}
\url{https://ko.wikipedia.org/wiki/%EC%84%9C%EB%B9%84%EC%8A%A4_%EA%B1%B0%EB%B6%80_%EA%B3%B5%EA%B2%A9 } 
\newline
\url{https://en.wikipedia.org/wiki/Denial-of-service_attack}
\newline
\url{https://security.radware.com/ddos-knowledge-center/ddos-attack-types/apdos-emerging-cyber-threat/}
\newline
\url{https://ko.wikipedia.org/wiki/SQL_%EC%82%BD%EC%9E%85}
\newline
\url{https://en.wikipedia.org/wiki/SQL_injection}
\newline
\url{https://ko.wikipedia.org/wiki/%EC%BD%94%EB%93%9C_%EC%9D%B8%EC%A0%9D%EC%85%98}
\newline
\url{https://ko.wikipedia.org/wiki/%EC%BB%B4%ED%93%A8%ED%84%B0_%EB%B0%94%EC%9D%B4%EB%9F%AC%EC%8A%A4}
\newline
\url{https://en.wikipedia.org/wiki/Computer_virus}
\newline
\url{https://ko.wikipedia.org/wiki/%EC%9B%9C}
\newline
\url{https://en.wikipedia.org/wiki/Computer_worm}
\newline
\url{https://ko.wikipedia.org/wiki/%EB%9E%9C%EC%84%AC%EC%9B%A8%EC%96%B4}
\newline
\url{https://ko.wikipedia.org/wiki/%EB%B4%87%EB%84%B7}
\newline
\url{https://ko.wikipedia.org/wiki/%ED%8A%B8%EB%A1%9C%EC%9D%B4_%EB%AA%A9%EB%A7%88_(%EC%BB%B4%ED%93%A8%ED%8C%85)}
\newline
\url{https://ko.wikipedia.org/wiki/%EC%82%AC%EB%AC%BC%EC%9D%B8%ED%84%B0%EB%84%B7}
\newline
\url{https://ko.wikipedia.org/wiki/%ED%81%B4%EB%9D%BC%EC%9A%B0%EB%93%9C_%EC%BB%B4%ED%93%A8%ED%8C%85}
\newline
\url{https://en.wikipedia.org/wiki/5G}
\newline
\url{https://ko.wikipedia.org/wiki/5%EC%84%B8%EB%8C%80_%EC%9D%B4%EB%8F%99_%ED%86%B5%EC%8B%A0}
\newline
\url{https://en.wikipedia.org/wiki/Small_cell}
\newline
\url{https://www.vodafone.qa/en/5g}

\end{document}
