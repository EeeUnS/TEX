
\chapter{Optimizing Program Performance}

\begin{enumerate}
    \item select an appropriate set of algorithms and data structures
    \item  write source code that the compiler can effectively optimize to turn into efficient executable code
    \item  divide a task into portions that can be computed in parallel, on some combination of multiple cores and multiple processors.
\end{enumerate}
명심할것. 두번째를 이해하기위해 컴파일러의 능력과 한계를 알아야한다. 최적화를 위하면서도 코드 가독성은 유지해야한다.


\section{Capabilities and Limitations of Optimizing Compilers}


컴파일러 최적화 명령어는 -0g -1g -2g ... 이 있다.


다음 두개의 코드가 있다고 생각해보자 
\begin{lstlisting}[style = CStyle]
void twiddle1(long *xp, long *yp)
{
*xp += *yp;
*xp += *yp;
}

void twiddle2(long *xp, long *yp)
{
*xp += 2* *yp;
}
\end{lstlisting}
twiddle1은 twiddle2로 최적화 될수있는가? 답은 아니 다. xp와 yp가 동일하다고 생각해보자. 그러면 명확할 것이다.


\begin{lstlisting}[style = CStyle]
long f();
long func1() {
return f() + f() + f() + f();
}

long func2() {
return 4*f();
}
\end{lstlisting}
func1 이 func2로 최적화 될 수 있다고 생각한다. 하지만 f에서 전역변수를 건든다고 생각해보자 4번 바뀔게 한번만 바뀌는 일이 될것이다.

컴파일러는 위험요소가 있을경우 최적화를 하지않는다.

\section{Eliminating Loop Inefficiencies}

\begin{lstlisting}[style = CStyle]
void combine1(vec_ptr v, data_t *dest)
 {
 long i;

 *dest = IDENT;
 for (i = 0; i < vec_length(v); i++) {
 data_t val;
 get_vec_element(v, i, &val);
 *dest = *dest OP val;
 }
 }

\end{lstlisting}

이게 어떻게 성능 개선이되는지 보자.

\begin{lstlisting}[style = CStyle]
void combine2(vec_ptr v, data_t *dest)
 {
 long i;
 long length = vec_length(v);

 *dest = IDENT;
 for (i = 0; i < length; i++) {
 data_t val;
 get_vec_element(v, i, &val);
 *dest = *dest OP val;
 }
 }
\end{lstlisting}


\begin{figure}[h!]
    \centering
    \includegraphics[scale=0.4]{pic/section5/pic1}
\end{figure}



\begin{lstlisting}[style = CStyle]
/* Convert string to lowercase: slow */
void lower1(char *s)
{
    long i;

    for (i = 0; i < strlen(s); i++)
        if (s[i] >= `A' && s[i] <= 'Z')
            s[i] -= ('A' - 'a');    
}

/* Convert string to lowercase: faster */
void lower2(char *s)
{
    long i;
    long len = strlen(s);

    for (i = 0; i < len; i++)
        if (s[i] >= `A' && s[i] <= 'Z')
            s[i] -= (`A' - 'a');
}

/* Sample implementation of library function strlen */
/* Compute length of string */
size_t strlen(const char *s)
{
    long length = 0;
    while (*s != '\0') {
        s++;
        length++;
    }
    return length;
}
\end{lstlisting}

\begin{figure}[h!]
    \centering
    \includegraphics[scale=0.4]{pic/section5/pic2}
    \caption{lower 성능비교}
\end{figure}

시간복잡도 계산으로도 충분히 알 수 있다.
문자열 길이가 변하는게 아니라면 strlen을 반복문안에 넣는 짓은 하지말자.

\section{Reducing Procedure Calls}

\begin{lstlisting}[style = CStyle]
void combine3(vec_ptr v, data_t *dest)
{
    long i;
    long length = vec_length(v);
    data_t *data = get_vec_start(v);
    *dest = IDENT;
    for (i = 0; i < length; i++) {
    *dest = *dest OP data[i];
    }
}
\end{lstlisting}

대충 함수안에서 계속 함수를 쳐부르는 짓은 오버헤드와 불리는 함수안에서 처리하는 불필요한 작업으로 느려진다는 뜻.
근데 뒤에 더다룬다함


\begin{figure}[h!]
    \centering
    \includegraphics[scale=0.3]{pic/section5/pic3}
\end{figure}

\section{Eliminating Unneeded Memory References}

combine3에서 내부 루프는 포인터 dest가 메모리 참조를 계속하는 식이다.
다음 방식이 조금더 효율적이다.

\begin{lstlisting}[style = CStyle]
void combine4(vec_ptr v, data_t *dest)
 {
 long i;
 long length = vec_length(v);
 data_t *data = get_vec_start(v);
 data_t acc = IDENT;

for (i = 0; i < length; i++) {
 acc = acc OP data[i];
 }
 *dest = acc;
 }
\end{lstlisting}




\begin{figure}[h!]
    \centering
    \includegraphics[scale=0.3]{pic/section5/pic4}
\end{figure}



\section{Understanding Modern Processors}

아몰랑


\section{Loop Unrolling}

while의 작동방식을 어셈블리로 한번보면 
if와  goto를 합쳐놓은 방식이다.
if는 단일 연산에비해느림 따라서
if검사를 적게 하게하면(=반복되는 횟수를 줄이면) 성능개선이 이루어진다.

\begin{lstlisting}[style = CStyle]
/* 2 x 1 loop unrolling */
void combine5(vec_ptr v, data_t *dest)
    {
    long i;
    long length = vec_length(v);
    long limit = length-1;
    data_t *data = get_vec_start(v);
    data_t acc = IDENT;

    /* Combine 2 elements at a time */
    for (i = 0; i < limit; i+=2) {
    acc = (acc OP data[i]) OP data[i+1];
    }

    /* Finish any remaining elements */
    for (; i < length; i++) {
    acc = acc OP data[i];
    }
    *dest = acc;
}
\end{lstlisting}


\begin{figure}[h!]
    \centering
    \includegraphics[scale=0.3]{pic/section5/pic5}
\end{figure}


\begin{figure}[h!]
    \centering
    \includegraphics[scale=0.4]{pic/section5/pic6}
    \caption{ kx1 에따른 성능비교}
\end{figure}


이 생각을 할수있다.
루프풀기를 최대로하면 제일 좋은게아닌가?

여러단점이있다.

\url{http://z3moon.com/프로그래밍/loop_unrolling}

\url{https://en.wikipedia.org/wiki/Loop_unrolling}

\begin{enumerate}
    \item 코드 크기 증가
    \item 가독성 저해
    \item 함수호출이 있을 경우 캐시 미스율 향상
\end{enumerate}

연산이 복잡해질수록 인덱스의 계산과 if조건이 수행시간에 영향을 주지않는다.
for문 내부가 간단한 코드일때 가장 효과가 좋다.

\section{Enhancing Parallelism}

\subsubsection{Multiple Accumulators}
연산을 나눠서 계산하고 마지막에 합치는 방식

\begin{lstlisting}[style = CStyle]
/* 2 x 2 loop unrolling */
void combine6(vec_ptr v, data_t *dest)
{
    long i;
    long length = vec_length(v);
    long limit = length-1;
    data_t *data = get_vec_start(v);
    data_t acc0 = IDENT;
    data_t acc1 = IDENT;

    /* Combine 2 elements at a time */
    for (i = 0; i < limit; i+=2) {
    acc0 = acc0 OP data[i];
    acc1 = acc1 OP data[i+1];
    }

    /* Finish any remaining elements */
    for (; i < length; i++) {
    acc0 = acc0 OP data[i];
    }
    *dest = acc0 OP acc1;
}
\end{lstlisting}

\begin{figure}[h!]
    \centering
    \includegraphics[scale=0.3]{pic/section5/pic7}
\end{figure}



\begin{figure}[h!]
    \centering
    \includegraphics[scale=0.3]{pic/section5/pic8}
    \caption{kxk}
\end{figure}


\subsubsection{Reassociation Transformation}


\begin{lstlisting}[style = CStyle]
/* 2 x 1a loop unrolling */
void combine7(vec_ptr v, data_t *dest)
{
long i;
long length = vec_length(v);
long limit = length-1;
data_t *data = get_vec_start(v);
data_t acc = IDENT;

/* Combine 2 elements at a time */
for (i = 0; i < limit; i+=2) {
acc = acc OP (data[i] OP data[i+1]);
}

/* Finish any remaining elements */
for (; i < length; i++) {
acc = acc OP data[i];
}
*dest = acc;
}
\end{lstlisting}

combine5의 다음코드를


\begin{lstlisting}[style = CStyle]
    acc = (acc OP data[i]) OP data[i+1];
\end{lstlisting}


\begin{lstlisting}[style = CStyle]
    acc = acc OP (data[i] OP data[i+1]);
\end{lstlisting}
로 바꾼다.

\begin{figure}[h!]
    \centering
    \includegraphics[scale=0.3]{pic/section5/pic7}
    \caption{}
\end{figure}



\begin{figure}[h!]
    \centering
    \includegraphics[scale=0.3]{pic/section5/pic8}
    \caption{}
\end{figure}


\section{Summary of Results for Optimizing Combining Code}

\section{Some Limiting Factors}
\subsection{Register Spilling}
    사용가능한 레지스터 수를 넘긴 병렬성을 갖는다면 값들을 메모리로 넘기는 식을 이용해서 런타임 스택에 공간을 할당한다.
\subsection{Branch Prediction and Misprediction Penalties}
    신경쓰지말것

\section{Understanding Memory Performance}
\subsection{Load Performance}
메모리 읽기에대해서 매 클럭사이클마다 유닛갯수만큼 읽을수있기 때문에 하한선이 존재한다.
\subsection{Store Performance}
store 유닛과 load 유닛의 데이터 의존성이 발생할 수 있다. 같은 주소를 바로 저장하고 읽는경우에 성능저하가 나올수있다. 이때 생성되는 연산이 7개의 클럭사이클이 필요.

\section{Life in the Real World: Performance Improvement Techniques}

\textrm{High-level design.} Choose appropriate algorithms and data structures for the problem at hand. Be especially vigilant to avoid algorithms or coding techniques that yield asymptotically poor performance.

\textrm{Basic coding principles.} Avoid optimization blockers so that a compiler can generate efficient code.
\begin{itemize}
    \item Eliminate excessive function calls. Move computations out of loops when possible. Consider selective compromises of program modularity to gain greater efficiency.
    \item Eliminate unnecessary memory references. Introduce temporary variables to hold intermediate results. Store a result in an array or global variable only when the final value has been computed.
\end{itemize}

\textrm{Low-level optimizations.} Structure code to take advantage of the hardware capabilities.
\begin{itemize}
    \item Unroll loops to reduce overhead and to enable further optimizations.
    \item Find ways to increase instruction-level parallelism by techniques such as multiple accumulators and reassociation.
    \item Rewrite conditional operations in a functional style to enable compilation via conditional data transfers.
\end{itemize}


5.14 Identifying and Eliminating Performance Bottlenecks 598


gprof

\begin{lstlisting}[language=bash]
    linux> gcc -Og -pg prog.c -o prog
    linux> ./prog file.txt
    linux> gprof prog
\end{lstlisting}

